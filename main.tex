\documentclass[aspectratio=169]{beamer}
\PassOptionsToPackage{english}{babel}
\usepackage{standardslides}

\usepackage{svg}

\usepackage{tikz}
\usepackage{pifont}
\usepackage{listings}
\usepackage{colortbl}
\newlength{\listingframemargin}
\setlength{\listingframemargin}{1em}
\newlength{\listingmargin}
\setlength{\listingmargin}{0.08\textwidth}

\definecolor{codeDarkGray}{gray}{0.2}
\definecolor{codeGray}{gray}{0.4}
\definecolor{codeLightGray}{rgb}{0.94,0.94,0.91}
\definecolor{codeBorder}{rgb}{0.34,0.24,0.21}
\definecolor{MidnightBlue}{rgb}{0.1, 0.1, 0.8}

\lstdefinestyle{standard}{%
  belowcaptionskip=0.5\baselineskip,
  breaklines=true,
  frameround=tttt,
  % frame=false,
  xleftmargin=0em,
  xrightmargin=0em,
  showstringspaces=false,
  showtabs=false,
  % tab=\smash{\rule[-.2\baselineskip]{.4pt}{\baselineskip}\kern.5em},
  basicstyle= \fontfamily{pcr}\selectfont\tiny\bfseries,
  keywordstyle= \bfseries\color{MidnightBlue}, %\color{codeDarkGray},
  commentstyle= \itshape\color{codeGray},
  identifierstyle=\color{codeDarkGray},
  stringstyle=\color{BurntOrange}, %\color{codeDarkGray},
  numberstyle=\tiny\ttfamily,
  % numbers=left,
  numbersep = 1em,
  % stepnumber = 1,
  % captionpos=t,
  tabsize=2,
  % backgroundcolor=\color{codebLightGray},
  rulecolor=\color{codeBorder},
  framexleftmargin=\listingframemargin,
  framexrightmargin=\listingframemargin
}

\newcommand{\inputCodeBlock}[1]{%
  % \begin{mybox}
    \begin{center}
      % \begin{minipage}[c]{0.7\textwidth}
        \lstinputlisting[%
          style = standard,
          language = c++,
          morekeywords={constexpr,noexcept,decltype,size_t,uint32_t,uint64_t,__m256i,__m256,__m256d,__m128i,__m128,__m128d}
        ]{#1}
      % \end{minipage}
    \end{center}
  % \end{mybox}
}

\def\UrlBigBreaks{\do\/\do-\do:}

\setbeamertemplate{footline}[frame number]
\setbeamertemplate{navigation symbols}{}

\title{%
  Probabilistic Circuits: \\ Marginal Maximum a Posteriori Queries%
}
% \subtitle{Master's Thesis Defense and Presentation}
\author{Markus Pawellek}

\bibliography{references}

\DeclareMathOperator*{\argmax}{arg\ max}
\DeclareMathOperator{\val}{val}
\DeclareMathOperator{\nodein}{in}

\begin{document}

\selectlanguage{english}

\frame[plain]{\titlepage}
\begin{frame}[plain]{Outline}
  \footnotesize
  \hfill\parbox[t][7cm][l]{0.9\textwidth}{\tableofcontents}
\end{frame}
\setcounter{framenumber}{0}

\section{Introduction}
  \begin{frame}{Introduction}
    \begin{itemize}
      \item Marginal maximum a posteriori (MMAP) combine marginal (MAR) and maximum a posteriori (MAP) inference
      \item We need sufficient conditions for tractability
      \item We use MAR and MAP algorithm
      \item Notations varies a little bit: no boldface letters
    \end{itemize}
  \end{frame}

\section{Background}
  \begin{frame}{Background}
    \begin{align*}
      X &\ldots \text{finite set of random variables} \\
      p\, &\ldots \text{joint probability distribution over $X$} \\
      \mathscr{C}\, &\ldots \text{PC over $X$ with $\mathscr{C}=(\mathscr{G},ϑ)$} \\
      Q &\ldots \text{set of query variables with $Q\subset X$}
    \end{align*}
  \end{frame}

  \begin{frame}{Background: MMAP Queries}
    \begin{itemize}
      \item query variables $Q$, evidence variables $E$, marginal variables $Z$ form partition of $X$
      \item $e\in E$, intevals $\mathscr{I}\subset\val(Z)$
    \end{itemize}
    \begin{mybox}
      \[
        \argmax_{q\in\val(Q)} \ p\roundBrackets{Q=q \ | \ E=e,Z\in\mathscr{I}}
      \]
      \[
        \argmax_{q\in\val(Q)} \integral{\mathscr{I}}{}{p(q,e,z)}{Z}
      \]
    \end{mybox}
  \end{frame}

  \begin{frame}{Background: MMAP Connection}
    \begin{mybox}
      \[
        \argmax_{q\in\val(Q)} \integral{\mathscr{I}}{}{p(q,e,z)}{Z}
      \]
    \end{mybox}
    \begin{itemize}
      \item $\mathbf{case}\ Q=\emptyset :\quad $ MAR Query
      \[
        \integral{\mathscr{I}}{}{p(e,z)}{Z}
      \]
      \item $\mathbf{case}\ Z=\emptyset :\quad $ MAP Query
      \[
        \argmax_{q\in\val(Q)}\ p(q,e)
      \]
    \end{itemize}
  \end{frame}

  \begin{frame}{Background: Review}
    \[
      \begin{aligned}[t]
        &\text{$\mathscr{C}$ is tractable for MAR queries} \\
        &\iff \text{$\mathscr{G}$ is decomposable and smooth}
      \end{aligned}
    \]
    \[
      \begin{aligned}[t]
        &\text{$\mathscr{C}$ is tractable for MAP queries} \\
        &\iff \text{$\mathscr{G}$ is consistent and deterministic}
      \end{aligned}
    \]
    \[
      \text{$\mathscr{G}$ is decomposable} \implies \text{$\mathscr{G}$ is consistent}
    \]
    \begin{mybox}
      \begin{align*}
        &\text{$\mathscr{C}$ is tractable for both MAR and MAP queries} \\
        &\iff \text{$\mathscr{G}$ is decomposable, smooth, and deterministic }
      \end{align*}
    \end{mybox}
  \end{frame}

\section{Marginal Determinism}
  \begin{frame}{Marginal Determinism}
    \begin{mybox}
      A sum node is marginal deterministic with respect to $Q$ if for all partial states $q\in Q$ at most one of its inputs is non-zero.
    \end{mybox}
    \begin{mybox}
      A circuit structure $\mathscr{G}$ is marginal deterministic with respect to $Q$ if for all sum units $n\in\mathscr{G}$ with $φ(n)\cap Q\neq\emptyset$, $n$ is marginal deterministic with respect to $Q$.
    \end{mybox}
    \begin{itemize}
      \item Simple generalization of determinism
      \item Structural property about the support of input units
      \item Defined with respect to $Q$
    \end{itemize}
  \end{frame}

  \begin{frame}{Marginal Determinism: Example}
    \center
    \includegraphics[width=0.9\textwidth]{figures/example.pdf}
  \end{frame}
  \begin{frame}{Marginal Determinism: Example}
    \center
    \includegraphics[height=0.8\textheight]{figures/big-example.pdf}
  \end{frame}

\section{Tractable Computation}
  \begin{frame}{Tractable Computation}
    \begin{itemize}
      \item Adjust algorithms for MAR and MAP queries
      \item Input units are assumed to provide correct output
      \item Product units are handled identically in MAR and MAP
      \item Sum units need a decision
      \item At the end, backward pass is needed to retrieve modes of input distributions
    \end{itemize}
    \begin{mybox}
      \[
        \begin{aligned}[t]
          \mathbf{if}\quad &φ(n) \cap Q \neq \emptyset \quad \mathbf{then} \\
          & r_n \longleftarrow \max_{c\in\nodein(n)} ϑ_{nc}r_c \\
          \mathbf{else} \\
          & r_n \longleftarrow \sum_{c\in\nodein(n)} ϑ_{nc}r_c
        \end{aligned}
      \]
    \end{mybox}
  \end{frame}

\section{Sufficient Conditions}
  \begin{frame}{Sufficient Conditions}
    \begin{mybox}
      \textbf{Theorem:}
      Let $\mathscr{G}$ be smooth, decomposable, and $Q$-marginal deterministic.
      Then for any parameterization $ϑ$ the above algorithm
      tractably computes MMAP queries of $\mathscr{C}$ over $Q$.
    \end{mybox}
    \begin{itemize}
      \item Proof by induction
      \item Input units are correct by assumption
    \end{itemize}
    \[
      \mathscr{Q}(e,\mathscr{I}) = \max_{q\in\val(Q)} \integral{\mathscr{I}}{}{\mathscr{C}(Z,q,e)}{Z}
    \]
  \end{frame}

  \begin{frame}{Sufficient Conditions: Product Units}
    \begin{itemize}
      \item Apply decomposability and partition $Z$, $Q$, and $E$
    \end{itemize}
    \begin{align*}
      \mathscr{Q}(e,\mathscr{I}) &= \max_{q_1,\ldots,q_k\in\val(Q)} \integral{\mathscr{I}}{}{\prod_{i=1}^k\mathscr{C}_i(Z_i,q_i,e_i)}{Z} \\
      &= \max_{q_1,\ldots,q_k\in\val(Q)} \prod_{i=1}^k \integral{\mathscr{I}}{}{\mathscr{C}_i(Z_i,q_i,e_i)}{Z_i} \\
      &= \prod_{i=1}^k \max_{q_i\in\val(Q)} \integral{\mathscr{I}_i}{}{\mathscr{C}_i(Z_i,q_i,e_i)}{Z_i} \\
      &= \prod_{i=1}^k \mathscr{C}_i(e_i,\mathscr{I}_i)
    \end{align*}
  \end{frame}

  \begin{frame}{Sufficient Conditions: Sum Units}
    \begin{itemize}
      \item Sum units with no query variable in their scope reduce to MAR queries which has alread been proven
      \item So $φ(n)\cap Q \neq \emptyset$
    \end{itemize}
    \begin{align*}
      \mathscr{Q}(e,\mathscr{I}) &= \max_{q\in\val(Q)} \integral{\mathscr{I}}{}{\sum_{i\in\nodein(n)} ϑ_i\mathscr{C}_i(Z,q,e)}{Z} \\
      &= \max_{q\in\val(Q)} \sum_{i\in\nodein(n)} \integral{\mathscr{I}}{}{ϑ_i\mathscr{C}_i(Z,q,e)}{Z} \\
      &= \max_{q\in\val(Q)} \max_{i\in\nodein(n)} \integral{\mathscr{I}}{}{ϑ_i\mathscr{C}_i(Z,q,e)}{Z} \\
      &= \max_{i\in\nodein(n)} ϑ_i \max_{q\in\val(Q)} \integral{\mathscr{I}}{}{\mathscr{C}_i(Z,q,e)}{Z} \\
      &= \max_{i\in\nodein(n)} ϑ_i \mathscr{Q}_i(e,\mathscr{I})
    \end{align*}
  \end{frame}

\section{Conclusions}
  \begin{frame}{Conclusions}
    \begin{itemize}
      \item MMAP queries are typically be NP-hard
      \item Complexity depends on set of query variables
      \item marginal determinism together with smoothness and decomposability seems to be sufficient for tractable computations
      \item sum units have to compute maxima when support contains query variables
    \end{itemize}
  \end{frame}

  \begin{frame}{Outlook}
    \begin{beamercolorbox}[sep=8pt,center,shadow=true,rounded=true]{title}
      \usebeamerfont{title}%
      Thank you for Your Attention!%
      \par%
    \end{beamercolorbox}
    \begin{itemize}
      \item Assuming marginal determinism for all sets of query variables strongly restricts the expressiveness of the probabilistic circuit
      \item More general approach by using sum-maximizer circuits
      \item No general algorithm how to generate one
      \item Tractable computation of information-theoretic measures, such as marginal entropy
    \end{itemize}
  \end{frame}

\setcounter{backupcounter}{\value{framenumber}}

\begin{frame}[plain]
  \frametitle{References}
  % \tiny
  % \AtNextBibliography{\tiny}
  % \begin{multicols}{2}
    \nocite{*}
    \printbibliography
  % \end{multicols}
\end{frame}

\setcounter{framenumber}{\value{backupcounter}}

\end{document}
